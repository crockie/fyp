\chapter{Introduction} \label{chapter:introduction}
\section{Background}

Automatic Speech Recognition (ASR) systems, which convert human speech into text, have become integral to modern voice-driven technologies. While current commercial ASR solutions excel in single-language environments, they often struggle with multilingual scenarios, due to inter- and intra-sentence language variety \cite{code_switching}. The research team at Nanyang Technological University (NTU) Speech Lab addresses this challenge through their innovative multilingual ASR model, capable of transcribing speech in English, Malay, Mandarin, and Singlish \cite{speech_lab,scdf_2}. This development is particularly significant in Singapore's context, where code-switching between languages and dialects is commonplace in daily communication.

One prominent user of this ASR system is the Singapore Civil Defence Force (SCDF), which leverages the live transcription service for their emergency call centers \cite{scdf}. The transcription system enables officers to record key information efficiently, saving critical time during emergencies. Consequently, the availability and reliability of the system are paramount, as any disruption could impede communication in life-saving scenarios.

The ASR system is currently deployed on Kubernetes across Microsoft Azure and Amazon Web Services (AWS) cloud platforms. Users interact with the system by establishing a WebSocket connection to the server, selecting their desired transcription model. The server assigns a worker to handle the task by forwarding audio data to the worker for transcription. The resulting transcript is then sent back to the user.

\section{Importance}
Previous Final Year Project (FYP) students have contributed to various aspects of this ASR system, such as deploying it on AWS with Terraform \cite{song_yu, kai_shern} and enhancing its security \cite{putra}. However, significant limitations persist. The direct communication between the server and workers results in a tightly coupled architecture, making it challenging to scale these components independently to meet fluctuating workloads \cite{tight_couple}. Additionally, both the server and workers maintain state information, such as audio data and worker statuses, exacerbating the scalability challenge.

System reliability is particularly affected by worker failures, which can occur due to various factors including resource exhaustion, node crashes, or network disruptions. These failures result in unprocessed audio segments, creating gaps in transcription that significantly impact service continuity. Such limitations directly compromise the system's ability to meet its Service Level Objectives (SLOs), particularly in terms of latency and availability, potentially affecting critical operations of its users.


\section{Objective, Scope, and Significance}
The objective of this FYP is to enhance the scalability and availability of the ASR system by transitioning to a decoupled architecture. To accomplish this, the project scope includes redesigning the system architecture to decouple components using a message queue; developing a dynamic and predictive scaling policy for workers; and designing mechanisms to minimize latency during worker failures. The modifications to the underlying ASR model, security aspects of the application and monitoring will be out-of-scope for this project.

The significance of this research lies in its potential to enhance the system’s fault tolerance and ability to handle fluctuating traffic loads. By addressing these limitations, the project aims to provide improved support for NTU Speech Lab’s users, ensuring the reliability of the ASR service for critical applications such as SCDF’s emergency operations.

\section{Report Organisation}
This report is structured into five chapters, each focusing on a specific aspect of the project:

\begin{itemize}
    \item \hyperref[chapter:introduction]{\textbf{Chapter 1: Introduction}} - This chapter provides an overview of the project, including its background, importance, objectives, scope, and significance.
    
    \item \hyperref[chapter:literature_review]{\textbf{Chapter 2: Literature Review}} - This chapter reviews previous FYP works on the ASR system and introduces the relevant technologies used in the project.
    
    \item \hyperref[chapter:analysis_and_design]{\textbf{Chapter 3: Analysis and Design}} - This chapter presents an analysis of the current ASR system architecture and identifies its limitations. It then describes the proposed decoupled architecture, including the use of message queues and dynamic scaling policies.
    
    \item \hyperref[chapter:detailed_implementation]{\textbf{Chapter 4: Detailed Implementation}} - This chapter provides a detailed implementation of the new architecture, the development of the dynamic scaling policy, and the mechanisms implemented to handle worker failures.
    
    \item \hyperref[chapter:conclusion_and_future_work]{\textbf{Chapter 5: Conclusion and Future Work}} - This chapter summarizes the contributions of the project. It also outlines potential areas for future research and development to further improve the ASR system.
\end{itemize}

