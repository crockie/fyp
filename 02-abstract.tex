\chapter*{Abstract}
\addcontentsline{toc}{chapter}{Abstract}

Automatic Speech Recognition (ASR) systems is an integral part of emergency call centers and other real-time transcription applications. To enhance the scalability and resilience of the ASR system, this project redesigned the existing architecture to a decoupled architecture. The new design leverages RabbitMQ for asynchronous messaging, enabling independent scaling of system components and improving fault tolerance. A dynamic scaling policy is implemented to adjust worker instances based on real-time workload demands, optimizing resource utilization and maintaining low-latency processing. Fault recovery mechanisms are introduced to minimize service disruptions and ensure continuous transcription tasks. The system is deployed on Kubernetes clusters hosted on Amazon Web Services (AWS), streamlining system management and enhancing scalability. Terraform is used for infrastructure-as-code to improve deployment reproducibility and operational efficiency. The project significantly enhances the system's scalability, resilience, and fault tolerance, benefiting Nanyang Technological University Speech Lab's research initiatives and critical applications such as Singapore Civil Defence Force's emergency call centers.
